\documentclass{article}
\usepackage{amsmath}
\usepackage{graphicx}

\begin{document}

\section{Scaling Laws}\label{sec:Scaling Laws}
The scaling laws used in this model are from \cite{housen2011ejecta}, which assume point-source crater evolution that applies to the final crater size, the growth of the transient crater and the majority of the observable ejecta field.

Depending on the size of the impact event, there are two separate cases: the strength regime and the gravity regime, discussed in Section \ref{ssec:Crater Size}. For smaller impacts, the strength regime dominates while for larger impacts, the gravity regime dominates. For materials that do not have a well defined strength (such as dry sand), the gravity regime dominates for all sizes of impacts.


%%%%%%%%%%%%%%%%%%%%%%%%%%%%%%%%%%%%%%%%%%%%%%%%%%%%%%%%%%%%%%%%%%%%%%
\subsection{Crater Size -- Strength \& Gravity Regime}\label{ssec:Crater Size}

The crater radius as determined by the \cite{housen2011ejecta} scaling laws is computed for both the strength and the gravity regime as the following:
\begin{equation}
R\left(\frac{\rho}{m}\right)^{1/3} = H_2\left(\frac{\rho}{\delta}\right)^{\frac{1-3\nu}{3}}\left[\frac{Y}{\rho U^2}\right]^{-\frac{\mu}{2}},
\end{equation}
for the strength regime with $R$ the crater radius, $\rho$ the target bulk density, $m$ the impactor mass, $\delta$ the impactor bulk density, $Y$ the material strength (shear for granular targets and tensile for solid targets), and $U$ the normal component\footnote{See Section 5.2 of \cite{housen2011ejecta}.} of the impactor speed. See Table \ref{tab:scaling law parameters} for various scaling law parameters, and
\begin{equation}
R\left(\frac{\rho}{m}\right)^{1/3} = H_1\left(\frac{\rho}{\delta}\right)^{\frac{2+\mu-6\nu}{3(2+\mu)}}\left[\frac{ga}{U^2}\right]^{-\frac{\mu}{2+\mu}},
\end{equation}
for the gravity regime with $g = GM/r_m^2 = 1.625$ m s$^{-2}$ the lunar surface gravity, and $a$ the impactor radius.  

%%%%%%%%%%%%%%%%%%%%%%%%%%%%%%%%%%%%%%%%%%%%%%%%%%%%%%%%%%%%%%%%%%%%%%%
%\subsection{Minimum \& Maximum Ejected Speed}\label{ssec:Min Max Ejecta Speed}
%
%%%%%%%%%%%%%%%%%%%%%%%%%%%%%%%%%%%%%%%%%%%%%%%%%%%%%%%%%%%%%%%%%%%%%%%
%\subsection{Maximum Ejected Particle Mass}

%%%%%%%%%%%%%%%%%%%%%%%%%%%%%%%%%%%%%%%%%%%%%%%%%%%%%%%%%%%%%%%%%%%%%%
\subsection{Mass Ejected from Crater}\label{ssec:Mass Ejected from Crater}

The mass ejected from a crater by an impact can be summarized into two equations parameterized by the position from the crater center $x$ as
\begin{align}
\frac{v}{U} &= C_1\left[\frac{x}{a}\left(\frac{\rho}{\delta}\right)^\nu\right]^{-\frac{1}{\mu}}\left(1 - \frac{x}{n_2 R}\right)^p,\\
\frac{M}{m} &= \frac{3k}{4\pi}\frac{\rho}{\delta}\left[\left(\frac{x}{a}\right)^3-n_1^3\right],
\end{align}
for $n_1 a \le x \le n_2 R$, where $v$ is the ejecta speed, and $M$ is the mass ejected at speeds equal to or greater than $v$.

The maximum speed occurs when $x = n_1 a$, or very close to ground zero, and is given by
\begin{equation}
\frac{v_{max}}{U} = C_1\left[\frac{x}{a}\left(\frac{\rho}{\delta}\right)^\nu\right]^{-\frac{1}{\mu}}\left(1 - \frac{n_1 a}{n_2 R}\right)^p.
\end{equation}
If $v_{max} \le 0$, then it is assumed no ejecta was created by the impact.

The total mass ejected is given when $x = n_2 R$,
\begin{equation}
\frac{M_{tot}}{m} = \frac{3k}{4\pi}\frac{\rho}{\delta}\left[\left(\frac{n_2 R}{a}\right)^3-n_1^3\right].
\end{equation}
It is equivalent to say that if $M_{tot} \le 0$, then no ejecta was created by the impact.



\end{document}
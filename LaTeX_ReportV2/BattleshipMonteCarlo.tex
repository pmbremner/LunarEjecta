\section{Battleship Monte Carlo}

The random sampling of the ejecta initial state (speed, zenith angle, and azimuth angle) to find a hit can be extremely laborious, especially when the asset is far from the point-of-impact. For simple cases, the set of hits can be found analytically. However, for the general case of complicated asset geometry or an asset that is above the lunar surface (e.g., in lunar orbit), it becomes difficult to solve these directly. Therefore, a Monte Carlo type method is needed to abstract away these complications.

In this section, the Battleship Monte Carlo method is introduced. Various ideas in the common literature are used, such as importance sampling and simulated annealing, but are done so in a dynamic way based on previously sampled points. The simple way Battleship Monte Carlo works is there are two strategies working together, a search method and a destroy method. The search method draws from a uniform distribution that covers the entire domain, for each dimension. Once hits are found, the destroy method kicks in to scan near previous hit locations, using a localized uniform distribution. As more hits are found, there are more choices for the destroy method to choose from in order to make a new shot. On average, the destroy method will find hits more efficiently than the search method, but may struggle with expanding to unknown territories. Hence, there should be a balance with searching and destroying. One possible technique to take advantage of both methods would be to start off with heavily searching and then over time switch to the destroy method once a sufficient number of hits are found over the entire domain in order to quickly fill in any valid hits (i.e., ejecta initial states).

The difficulty with changing the probability distribution function (PDF) every iteration, in terms of importance sampling, is that there needs to be a concise way to track this PDF and keep it normalized correctly.